\documentclass[a4paper]{article}

%use the english line for english reports
%usepackage[english]{babel}
\usepackage[portuguese]{babel}
\usepackage[utf8]{inputenc}
\usepackage{indentfirst}
\usepackage{graphicx}
\usepackage{verbatim}


\begin{document}

\setlength{\textwidth}{16cm}
\setlength{\textheight}{22cm}

\title{\Huge\textbf{Protocolo de Ligação de Dados}\linebreak\linebreak\linebreak
\Large\textbf{Relatório}\linebreak\linebreak
\linebreak\linebreak
\includegraphics[scale=0.1]{feup-logo.png}\linebreak\linebreak
\linebreak\linebreak
\Large{Mestrado Integrado em Engenharia Informática e Computação} \linebreak\linebreak
\Large{Redes de Computadores}\linebreak
}

\author{\textbf{Grupo xx:}\\
José Carlos Alves Vieira - up2014 \\
Renato Sampaio de Abreu - up201403377 \\
\linebreak\linebreak \\
 \\ Faculdade de Engenharia da Universidade do Porto \\ Rua Roberto Frias, s\/n, 4200-465 Porto, Portugal \linebreak\linebreak\linebreak
\linebreak\linebreak\vspace{1cm}}

\maketitle
\thispagestyle{empty}

%************************************************************************************************
%************************************************************************************************

\newpage

%Todas as figuras devem ser referidas no texto. %\ref{fig:codigoFigura}
%
%%Exemplo de código para inserção de figuras
%%\begin{figure}[h!]
%%\begin{center}
%%escolher entre uma das seguintes três linhas:
%%\includegraphics[height=20cm,width=15cm]{path relativo da imagem}
%%\includegraphics[scale=0.5]{path relativo da imagem}
%%\includegraphics{path relativo da imagem}
%%\caption{legenda da figura}
%%\label{fig:codigoFigura}
%%\end{center}
%%\end{figure}
%
%
%\textit{Para escrever em itálico}
%\textbf{Para escrever em negrito}
%Para escrever em letra normal
%``Para escrever texto entre aspas''
%
%Para fazer parágrafo, deixar uma linha em branco.
%
%Como fazer bullet points:
%\begin{itemize}
	%\item Item1
	%\item Item2
%\end{itemize}
%
%Como enumerar itens:
%\begin{enumerate}
	%\item Item 1
	%\item Item 2
%\end{enumerate}
%
%\begin{quote}``Isto é uma citação''\end{quote}


%%%%%%%%%%%%%%%%%%%%%%%%%%
\section{Introdução}

Indicação dos objectivos do trabalho e do relatório; descrição da lógica do relatório com indicações sobre o tipo de informação que poderá ser encontrada em cada uma secções seguintes.


%%%%%%%%%%%%%%%%%%%%%%%%%%
\section{Arquitetura}

Blocos funcionais e interfaces.

%%%%%%%%%%%%%%%%%%%%%%%%%%
\subsection{Estrutura do código}

APIs, principais estruturas de dados, principais funções e sua relação com a arquitetura

%%%%%%%%%%%%%%%%%%%%%%%%%%
\subsection{Casos de uso principais}

Elencar os movimentos (tipos de jogadas) possíveis e definir os cabeçalhos dos predicados que serão utilizados (ainda não precisam de estar implementados).

%%%%%%%%%%%%%%%%%%%%%%%%%%
\section{Protocolo de ligação lógica}

Identificação dos principais aspectos funcionais; descrição da estratégia de implementação destes aspectos com apresentação de extratos de código.

%%%%%%%%%%%%%%%%%%%%%%%%%%
\section{Protocolo de aplicação}

Identificação dos principais aspectos funcionais; descrição da estratégia de implementação destes aspectos com apresentação de extratos de código.

%%%%%%%%%%%%%%%%%%%%%%%%%%
\section{Elementos de valorização}

Identificação dos elementos de valorização implementados; descrição da estratégia de implementação com apresentação de pequenos extratos de código.

%%%%%%%%%%%%%%%%%%%%%%%%%%
\section{Conclusões}

Síntese da informação apresentada nas secções anteriores; reflexão sobre os objectivos de aprendizagem alcançados



\end{document}
